% LaTeX file for resume 
% This file uses the resume document class (res.cls)

% all city - [1] 19.20 25.00 49.22, 2026

\documentclass{res} 
\usepackage[landscape]{geometry}
\usepackage{array}
\usepackage[usenames,dvipsnames]{xcolor}
\usepackage{enumerate}

%\usepackage{helvetica} % uses helvetica postscript font (download helvetica.sty)
%\usepackage{newcent}   % uses new century schoolbook postscript font 
\newsectionwidth{8pt}  % So the text is not indented under section headings
\setlength{\textheight}{10.2in} % set text height big enough for box
\topmargin=-.80in       % to start box .5in from top of page
\oddsidemargin=-.85in   % to start box .5in from left of page
\newcommand{\tab}{\hspace*{2em}}

\begin{document}

%%%%%%%%%%%%%%%%%%%%%%%%%%%%%%%%%%%%%%%%%%%%%%%%%%%%%%%%%%%%%%%%%%%%%%%%%%%%
% The following lines define \boxaround, used to draw a box on the page.
% The parameter is the entire text of the resume. Must fit on one page!
%
% \boxaroundhmargin is the left & right margin around the text inside the box.
% \boxaroundvmargin is the top & bottom margin around the text inside the box.
% \boxrulethickness controls thickness of line used to draw the box.
% You can change these 3 things in the lines below:
%%%%%%%%%%%%%%%%%%%%%%%%%%%%%%%%%%%%%%%%%%%%%%%%%%%%%%%%%%%%%%%%%%%%%%%%%%%%%
\newdimen\boxrulethickness\newdimen\boxarounsedhmargin\newdimen\boxaroundvmargin
\boxrulethickness=2pt        %controls thickness of line 
%\boxaroundhmargin=30pt        % about a half inch
\boxaroundvmargin=2pt        % to fit more text on page, make this smaller
%%%%%%%%%%%%%%%%%%%%%%%%% Don't read this stuff %%%%%%%%%%%%%%%%%%%%%%%%%%%%%%
\hsize=10.5in \vsize=8.5in             % use bigger dimensions for box
\newbox\MACboxA  \newdimen\MACdimenA
% \borderandboxit is used inside \boxaround:
\def\borderandboxit#1#2#3{\vbox{\hrule height#2\hbox{\vrule width#2\hskip#1\hskip-#2%
  \vbox{\vskip#1\relax#3\vskip#1}\hskip#1\hskip-#2\vrule width#2}\hrule height#2}}
%
\long\def\boxaround#1{\vskip6pt
  {\MACdimenA=\hsize \advance\MACdimenA by-\boxaroundhmargin
   \advance\MACdimenA by-\boxaroundhmargin   % once for each side
   \setbox\MACboxA=\hbox to \hsize{\hskip\boxaroundhmargin%\hss
                     \vbox{\hsize=\MACdimenA
                           \vskip\boxaroundvmargin #1
                           \vskip\boxaroundvmargin}\hss}%
   \borderandboxit{0pt}\boxrulethickness{\box\MACboxA}}%
  \vskip2pt plus0pt minus0pt
}
%%%%%%%%%%%%%%%%%%%  End of \boxaround macro %%%%%%%%%%%%%%%%%%%%%%%%%%%%%%%%%
\boxaround{% put the text on the page inside a box  
\smallskip
\centerline{\LARGE \bf Modeling a Sampling Problem}
\smallskip
\smallskip
\centerline{\large Steven Ellis -- Company X Research}

\vspace{-17mm}
\begin{tabular}{c}
{
\hspace{-3mm}
\pdfximage
width 26 cm {model.pdf}
\pdfrefximage
\pdflastximage
}
\end{tabular}

\begin{center}
\vspace{-10mm}
\line(1,0){35mm}
\end{center}
\begin{center}
\begin{tabular}{>{\centering\arraybackslash}p{14cm}p{}>{\centering\arraybackslash}p{10cm}}
\vspace{-47mm}
\begin{minipage}{14cm}
\section{Server Load \& Request Allotment}
Load balancing is a computer networking method for distributing workloads across multiple computing resources, such as computers, a computer cluster, network links, central processing units or disk drives. Load balancing aims to optimize resource use, maximize throughput, minimize response time, and avoid overload of any one of the resources. Using multiple components with load balancing instead of a single component may increase reliability through redundancy. Load balancing is usually provided by dedicated software or hardware, such as a multilayer switch or a Domain Name System server process.
\section{Operational Situation}
The heroku.com stack only supports single threaded requests. Even if your application were to fork and support handling multiple requests at once, the routing mesh will never serve more than a single request to a dyno at a time. A load-balancer can manage 288 dynes, blocked into groups of 32. Given 32 dynos, new requests are randomly assigned to each dyno.
\section{SRSWOR}
We iterate along these ranges (1:32, 33:65) to see how common a given persist load is likely to be within each load-balancer. Simple Random Sampling Without Replacement (SRSWOR) models situations where individuals, once picked as part of a sample from a population, are not returned to the population for potential re-sampling.
\end{minipage}
&
{}&
{
\begin{minipage}{10cm}
\begin{center}
\begin{tabular}{| l | llllll |}
\hline
\multicolumn{7}{|c|}
	{\rule[-2mm]{0mm}{6mm}Predicted Server Load} \\
\hline
Persists& Min &1st Q. & Median & Mean &3rd Q.& Max \\
\hline		
10\% & 0 &   2 &   3 &   3.2 & 4 &   11 \\ 
20\% &1 &   5 &   6 &   6.4 & 8 &   14 \\ 
30\% &1 &   8 &   10 &  9.6 & 11 &  20 \\ 
40\% &4 &    11 &   13 &   12.8 & 15 &   22 \\  
50\% &7 &  14 & 16 & 16 & 18 & 26 \\
60\% &10 &   17 &   19 &   19.2 & 21 &   30 \\  
70\% &13 &   21 &   23 &   22.4 & 24 &   30 \\  
80\% &17 &   24 &   26 &   25.6 & 27 &   32 \\
90\% &22 &   28 &   29 &   28.8 & 30 &   32 \\
\hline
\multicolumn{7}{c}{} \\

\hline
\multicolumn{7}{|c|}
	{\linewidth{}\rule[-2mm]{0mm}{6mm}Server Options\footnote[1]{BW = bandwidth in mbps}} \\
\hline
\multicolumn{3}{|l|}{Persists per server}&	32&	24&	16&	Pt.-to-Pt.\\
\hline
\multicolumn{3}{|l|}{Model A } &	x	&y	&z	&q\\
\multicolumn{3}{|l|}{Model B }&	x	&y	&z	&q\\
\multicolumn{3}{|l|}{Model C }&	x	&y	&z	&q\\
\multicolumn{3}{|l|}{Model D }&	x	&y	&z	&q\\
\hline

\end{tabular}
\end{center}
}
\end{minipage}
\end{tabular}
\end{center}

\vfill} %    end the material being boxed.
\end{document}







